% #######################################
% ########### FILL THESE IN #############
% #######################################
\def\mytitle{This is a title}
\def\mykeywords{Fill, These, In, So, google, can, find, your, report}
\def\myauthor{Miles Budden C1769331}
\def\contact{buddenm@cardiff.ac.uk}
\def\mymodule{Module Name (CM9999)}
% #######################################
% #### YOU DON'T NEED TO TOUCH BELOW ####
% #######################################
\documentclass[10pt, a4paper]{article}
\usepackage[a4paper,outer=1.5cm,inner=1.5cm,top=1.75cm,bottom=1.5cm]{geometry}
\twocolumn
\usepackage{graphicx}
\graphicspath{{./images/}}
%colour our links, remove weird boxes
\usepackage[hyphens]{url}
\usepackage[hidelinks]{hyperref}
%Stop indentation on new paragraphs
\usepackage[parfill]{parskip}
\usepackage{amssymb}

\usepackage{pgfplots}




%Cardiff logo top right
\usepackage{watermark}
%Lorem Ipusm dolor please don't leave any in you final report ;)
\usepackage{lipsum}
\usepackage{xcolor}
\usepackage{listings}
\usepackage{minted}
\setminted{
    frame=lines,
    framesep=2mm,
    baselinestretch=1.2,
    fontsize=\footnotesize,
    breaklines
}
%give us the Capital H that we all know and love
\usepackage{float}
%tone down the line spacing after section titles
\usepackage{titlesec}
%Cool maths printing
\usepackage{amsmath}
%PseudoCode
\usepackage{algorithm2e}
\usepackage{natbib}

\titlespacing{\subsection}{0pt}{\parskip}{-3pt}
\titlespacing{\subsubsection}{0pt}{\parskip}{-\parskip}
\titlespacing{\paragraph}{0pt}{\parskip}{\parskip}
\newcommand{\figuremacro}[5]{
    \begin{figure}[#1]
        \centering
        \includegraphics[width=#5\columnwidth]{#2}
        \caption[#3]{\textbf{#3}#4}
        \label{fig:#2}
    \end{figure}
}


\thiswatermark{\centering \put(440.5,-80.0){\includegraphics[scale=0.15]{logo}} }
\title{\mytitle}
\author{\myauthor\hspace{1em}\\\contact\\Cardiff University\hspace{0.5em}-\hspace{0.5em}\mymodule}
\date{}
\hypersetup{pdfauthor=\myauthor,pdftitle=\mytitle,pdfkeywords=\mykeywords}
\sloppy
% #######################################
% ########### START FROM HERE ###########
% #######################################
\begin{document}
\maketitle
\tableofcontents
\section{Introduction}
\lipsum[2]
\begin{algorithm}[h]
\DontPrintSemicolon
\KwData{An array, $A$, of length $n$}
\KwResult{An array, $A$, sorted}
$limit \leftarrow n - 1$\;
$done \leftarrow False$\;
\While{$\neg\,done$}{
    $done \leftarrow True$\;
    \For{$j \leftarrow 1$ \KwTo $limit$}{
        \If{$A[j+1]<A[j]$}{
            $temp \leftarrow A[j]$\;
            $A[j] \leftarrow A[j+1]$\;
            $A[j + 1] \leftarrow temp$\;
            $done \leftarrow False$\;
        }
    }
    $limit \leftarrow limit - 1$\;
}
\label{alg:bubbleimp}
\caption{Blah blah \citep{Cormen1990}}
\end{algorithm}

\subsection{Other stuff}
\lipsum[1]
\lipsum[3]


\section{More stuff}
\subsection{Yessss}
\lipsum[1]
\lipsum[3]

\begin{minted}
{java}
/**
 * Loads a data file and returns a parsed list of unique words longer than 3
 * characters with no capitals and no punctuation.
 *
 * @param path The path to the data file
 * @param length The number of words that shall be fetched
 *
 * @return The parsed list of words
 *
 * @throws FileNotFoundException Thrown when the data file cannot be loaded
 *
 * @see main
 */
public static List<String> parseData(String path, int length) throws FileNotFoundException{
    // List to store the words in
    List<String> words = new ArrayList<String>();
    try {
        // Open and read the file
        File fileIn = new File(path);
        Scanner in = new Scanner(fileIn);

        // Iterate through the lines
        while (in.hasNext()) {
            // Get a line of the text file with no punctuation or capital letters
            String word = in.next().replaceAll("[^a-zA-Z]", "").toLowerCase();
            // Add the word to the list if it isn't already there and if it is longer than 3 characters
            if (!words.contains(word) && word.length() > 3) {
                words.add(word);
            }
            // Stop adding words when enough have been found
            if (words.size() >= length) {
                break;
            }
        }
        // Finish up
        in.close();
        return words;
    } catch (FileNotFoundException e) {
        throw e;
    }
}
\end{minted}

\bibliographystyle{cardiff}
\bibliography{references}
		
\end{document}